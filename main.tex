% This is a template for your written document.
%
% To compile using latexmk on the command line, run the following: 
% latexmk -pdf main.tex

\documentclass[12pt]{article}
\usepackage{setspace}
\singlespace
\usepackage[left=1in,right=1in,top=1in,bottom=1in]{geometry}
\usepackage{graphicx}

\title{\textbf{Project Topic}}
\author{Your Name}

\begin{document}

\maketitle

\section{Introduction}
The introduction will broadly introduce your topic, motivate it, and describe what will be done.

\section{Related Works}

\section{Methodology}
Dijkstra's algorithm was chosen for single-source shortest path computation on graphs with non-negative edge weights, following the implementation presented in CLRS~\cite{clrsAlgorithms}.

\section{Results}
An exemplary figure is shown in Figure~\ref{fig:dijkstra}. Any time a figure is used, it must have a capation and be directly referred to and discussed in the text.

%\begin{figure}[b] puts the image to the (closest) bottom of page
%\begin{figure}[t] puts the image to the (closest) top of page
%\begin{figure}[h] allows the image to float, putting it roughly wherever it was placed in the text.
\begin{figure}[b] 
\begin{center}
  \includegraphics[width=5cm]{imgs/dijkstra.png}
  \caption{Dijkstra's algorithm does not work on graphs with negative edge weights, as evidenced here.}
  \label{fig:dijkstra}
 \end{center}
\end{figure}



\section{Conclusion}

\bibliographystyle{acm}
\bibliography{bibliography.bib}

\end{document}
